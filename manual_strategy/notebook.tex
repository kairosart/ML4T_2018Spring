
% Default to the notebook output style

    


% Inherit from the specified cell style.




    
\documentclass[11pt]{article}

    
    
    \usepackage[T1]{fontenc}
    % Nicer default font (+ math font) than Computer Modern for most use cases
    \usepackage{mathpazo}

    % Basic figure setup, for now with no caption control since it's done
    % automatically by Pandoc (which extracts ![](path) syntax from Markdown).
    \usepackage{graphicx}
    % We will generate all images so they have a width \maxwidth. This means
    % that they will get their normal width if they fit onto the page, but
    % are scaled down if they would overflow the margins.
    \makeatletter
    \def\maxwidth{\ifdim\Gin@nat@width>\linewidth\linewidth
    \else\Gin@nat@width\fi}
    \makeatother
    \let\Oldincludegraphics\includegraphics
    % Set max figure width to be 80% of text width, for now hardcoded.
    \renewcommand{\includegraphics}[1]{\Oldincludegraphics[width=.8\maxwidth]{#1}}
    % Ensure that by default, figures have no caption (until we provide a
    % proper Figure object with a Caption API and a way to capture that
    % in the conversion process - todo).
    \usepackage{caption}
    \DeclareCaptionLabelFormat{nolabel}{}
    \captionsetup{labelformat=nolabel}

    \usepackage{adjustbox} % Used to constrain images to a maximum size 
    \usepackage{xcolor} % Allow colors to be defined
    \usepackage{enumerate} % Needed for markdown enumerations to work
    \usepackage{geometry} % Used to adjust the document margins
    \usepackage{amsmath} % Equations
    \usepackage{amssymb} % Equations
    \usepackage{textcomp} % defines textquotesingle
    % Hack from http://tex.stackexchange.com/a/47451/13684:
    \AtBeginDocument{%
        \def\PYZsq{\textquotesingle}% Upright quotes in Pygmentized code
    }
    \usepackage{upquote} % Upright quotes for verbatim code
    \usepackage{eurosym} % defines \euro
    \usepackage[mathletters]{ucs} % Extended unicode (utf-8) support
    \usepackage[utf8x]{inputenc} % Allow utf-8 characters in the tex document
    \usepackage{fancyvrb} % verbatim replacement that allows latex
    \usepackage{grffile} % extends the file name processing of package graphics 
                         % to support a larger range 
    % The hyperref package gives us a pdf with properly built
    % internal navigation ('pdf bookmarks' for the table of contents,
    % internal cross-reference links, web links for URLs, etc.)
    \usepackage{hyperref}
    \usepackage{longtable} % longtable support required by pandoc >1.10
    \usepackage{booktabs}  % table support for pandoc > 1.12.2
    \usepackage[inline]{enumitem} % IRkernel/repr support (it uses the enumerate* environment)
    \usepackage[normalem]{ulem} % ulem is needed to support strikethroughs (\sout)
                                % normalem makes italics be italics, not underlines
    

    
    
    % Colors for the hyperref package
    \definecolor{urlcolor}{rgb}{0,.145,.698}
    \definecolor{linkcolor}{rgb}{.71,0.21,0.01}
    \definecolor{citecolor}{rgb}{.12,.54,.11}

    % ANSI colors
    \definecolor{ansi-black}{HTML}{3E424D}
    \definecolor{ansi-black-intense}{HTML}{282C36}
    \definecolor{ansi-red}{HTML}{E75C58}
    \definecolor{ansi-red-intense}{HTML}{B22B31}
    \definecolor{ansi-green}{HTML}{00A250}
    \definecolor{ansi-green-intense}{HTML}{007427}
    \definecolor{ansi-yellow}{HTML}{DDB62B}
    \definecolor{ansi-yellow-intense}{HTML}{B27D12}
    \definecolor{ansi-blue}{HTML}{208FFB}
    \definecolor{ansi-blue-intense}{HTML}{0065CA}
    \definecolor{ansi-magenta}{HTML}{D160C4}
    \definecolor{ansi-magenta-intense}{HTML}{A03196}
    \definecolor{ansi-cyan}{HTML}{60C6C8}
    \definecolor{ansi-cyan-intense}{HTML}{258F8F}
    \definecolor{ansi-white}{HTML}{C5C1B4}
    \definecolor{ansi-white-intense}{HTML}{A1A6B2}

    % commands and environments needed by pandoc snippets
    % extracted from the output of `pandoc -s`
    \providecommand{\tightlist}{%
      \setlength{\itemsep}{0pt}\setlength{\parskip}{0pt}}
    \DefineVerbatimEnvironment{Highlighting}{Verbatim}{commandchars=\\\{\}}
    % Add ',fontsize=\small' for more characters per line
    \newenvironment{Shaded}{}{}
    \newcommand{\KeywordTok}[1]{\textcolor[rgb]{0.00,0.44,0.13}{\textbf{{#1}}}}
    \newcommand{\DataTypeTok}[1]{\textcolor[rgb]{0.56,0.13,0.00}{{#1}}}
    \newcommand{\DecValTok}[1]{\textcolor[rgb]{0.25,0.63,0.44}{{#1}}}
    \newcommand{\BaseNTok}[1]{\textcolor[rgb]{0.25,0.63,0.44}{{#1}}}
    \newcommand{\FloatTok}[1]{\textcolor[rgb]{0.25,0.63,0.44}{{#1}}}
    \newcommand{\CharTok}[1]{\textcolor[rgb]{0.25,0.44,0.63}{{#1}}}
    \newcommand{\StringTok}[1]{\textcolor[rgb]{0.25,0.44,0.63}{{#1}}}
    \newcommand{\CommentTok}[1]{\textcolor[rgb]{0.38,0.63,0.69}{\textit{{#1}}}}
    \newcommand{\OtherTok}[1]{\textcolor[rgb]{0.00,0.44,0.13}{{#1}}}
    \newcommand{\AlertTok}[1]{\textcolor[rgb]{1.00,0.00,0.00}{\textbf{{#1}}}}
    \newcommand{\FunctionTok}[1]{\textcolor[rgb]{0.02,0.16,0.49}{{#1}}}
    \newcommand{\RegionMarkerTok}[1]{{#1}}
    \newcommand{\ErrorTok}[1]{\textcolor[rgb]{1.00,0.00,0.00}{\textbf{{#1}}}}
    \newcommand{\NormalTok}[1]{{#1}}
    
    % Additional commands for more recent versions of Pandoc
    \newcommand{\ConstantTok}[1]{\textcolor[rgb]{0.53,0.00,0.00}{{#1}}}
    \newcommand{\SpecialCharTok}[1]{\textcolor[rgb]{0.25,0.44,0.63}{{#1}}}
    \newcommand{\VerbatimStringTok}[1]{\textcolor[rgb]{0.25,0.44,0.63}{{#1}}}
    \newcommand{\SpecialStringTok}[1]{\textcolor[rgb]{0.73,0.40,0.53}{{#1}}}
    \newcommand{\ImportTok}[1]{{#1}}
    \newcommand{\DocumentationTok}[1]{\textcolor[rgb]{0.73,0.13,0.13}{\textit{{#1}}}}
    \newcommand{\AnnotationTok}[1]{\textcolor[rgb]{0.38,0.63,0.69}{\textbf{\textit{{#1}}}}}
    \newcommand{\CommentVarTok}[1]{\textcolor[rgb]{0.38,0.63,0.69}{\textbf{\textit{{#1}}}}}
    \newcommand{\VariableTok}[1]{\textcolor[rgb]{0.10,0.09,0.49}{{#1}}}
    \newcommand{\ControlFlowTok}[1]{\textcolor[rgb]{0.00,0.44,0.13}{\textbf{{#1}}}}
    \newcommand{\OperatorTok}[1]{\textcolor[rgb]{0.40,0.40,0.40}{{#1}}}
    \newcommand{\BuiltInTok}[1]{{#1}}
    \newcommand{\ExtensionTok}[1]{{#1}}
    \newcommand{\PreprocessorTok}[1]{\textcolor[rgb]{0.74,0.48,0.00}{{#1}}}
    \newcommand{\AttributeTok}[1]{\textcolor[rgb]{0.49,0.56,0.16}{{#1}}}
    \newcommand{\InformationTok}[1]{\textcolor[rgb]{0.38,0.63,0.69}{\textbf{\textit{{#1}}}}}
    \newcommand{\WarningTok}[1]{\textcolor[rgb]{0.38,0.63,0.69}{\textbf{\textit{{#1}}}}}
    
    
    % Define a nice break command that doesn't care if a line doesn't already
    % exist.
    \def\br{\hspace*{\fill} \\* }
    % Math Jax compatability definitions
    \def\gt{>}
    \def\lt{<}
    % Document parameters
    \title{manual\_strategy}
    
    
    

    % Pygments definitions
    
\makeatletter
\def\PY@reset{\let\PY@it=\relax \let\PY@bf=\relax%
    \let\PY@ul=\relax \let\PY@tc=\relax%
    \let\PY@bc=\relax \let\PY@ff=\relax}
\def\PY@tok#1{\csname PY@tok@#1\endcsname}
\def\PY@toks#1+{\ifx\relax#1\empty\else%
    \PY@tok{#1}\expandafter\PY@toks\fi}
\def\PY@do#1{\PY@bc{\PY@tc{\PY@ul{%
    \PY@it{\PY@bf{\PY@ff{#1}}}}}}}
\def\PY#1#2{\PY@reset\PY@toks#1+\relax+\PY@do{#2}}

\expandafter\def\csname PY@tok@w\endcsname{\def\PY@tc##1{\textcolor[rgb]{0.73,0.73,0.73}{##1}}}
\expandafter\def\csname PY@tok@c\endcsname{\let\PY@it=\textit\def\PY@tc##1{\textcolor[rgb]{0.25,0.50,0.50}{##1}}}
\expandafter\def\csname PY@tok@cp\endcsname{\def\PY@tc##1{\textcolor[rgb]{0.74,0.48,0.00}{##1}}}
\expandafter\def\csname PY@tok@k\endcsname{\let\PY@bf=\textbf\def\PY@tc##1{\textcolor[rgb]{0.00,0.50,0.00}{##1}}}
\expandafter\def\csname PY@tok@kp\endcsname{\def\PY@tc##1{\textcolor[rgb]{0.00,0.50,0.00}{##1}}}
\expandafter\def\csname PY@tok@kt\endcsname{\def\PY@tc##1{\textcolor[rgb]{0.69,0.00,0.25}{##1}}}
\expandafter\def\csname PY@tok@o\endcsname{\def\PY@tc##1{\textcolor[rgb]{0.40,0.40,0.40}{##1}}}
\expandafter\def\csname PY@tok@ow\endcsname{\let\PY@bf=\textbf\def\PY@tc##1{\textcolor[rgb]{0.67,0.13,1.00}{##1}}}
\expandafter\def\csname PY@tok@nb\endcsname{\def\PY@tc##1{\textcolor[rgb]{0.00,0.50,0.00}{##1}}}
\expandafter\def\csname PY@tok@nf\endcsname{\def\PY@tc##1{\textcolor[rgb]{0.00,0.00,1.00}{##1}}}
\expandafter\def\csname PY@tok@nc\endcsname{\let\PY@bf=\textbf\def\PY@tc##1{\textcolor[rgb]{0.00,0.00,1.00}{##1}}}
\expandafter\def\csname PY@tok@nn\endcsname{\let\PY@bf=\textbf\def\PY@tc##1{\textcolor[rgb]{0.00,0.00,1.00}{##1}}}
\expandafter\def\csname PY@tok@ne\endcsname{\let\PY@bf=\textbf\def\PY@tc##1{\textcolor[rgb]{0.82,0.25,0.23}{##1}}}
\expandafter\def\csname PY@tok@nv\endcsname{\def\PY@tc##1{\textcolor[rgb]{0.10,0.09,0.49}{##1}}}
\expandafter\def\csname PY@tok@no\endcsname{\def\PY@tc##1{\textcolor[rgb]{0.53,0.00,0.00}{##1}}}
\expandafter\def\csname PY@tok@nl\endcsname{\def\PY@tc##1{\textcolor[rgb]{0.63,0.63,0.00}{##1}}}
\expandafter\def\csname PY@tok@ni\endcsname{\let\PY@bf=\textbf\def\PY@tc##1{\textcolor[rgb]{0.60,0.60,0.60}{##1}}}
\expandafter\def\csname PY@tok@na\endcsname{\def\PY@tc##1{\textcolor[rgb]{0.49,0.56,0.16}{##1}}}
\expandafter\def\csname PY@tok@nt\endcsname{\let\PY@bf=\textbf\def\PY@tc##1{\textcolor[rgb]{0.00,0.50,0.00}{##1}}}
\expandafter\def\csname PY@tok@nd\endcsname{\def\PY@tc##1{\textcolor[rgb]{0.67,0.13,1.00}{##1}}}
\expandafter\def\csname PY@tok@s\endcsname{\def\PY@tc##1{\textcolor[rgb]{0.73,0.13,0.13}{##1}}}
\expandafter\def\csname PY@tok@sd\endcsname{\let\PY@it=\textit\def\PY@tc##1{\textcolor[rgb]{0.73,0.13,0.13}{##1}}}
\expandafter\def\csname PY@tok@si\endcsname{\let\PY@bf=\textbf\def\PY@tc##1{\textcolor[rgb]{0.73,0.40,0.53}{##1}}}
\expandafter\def\csname PY@tok@se\endcsname{\let\PY@bf=\textbf\def\PY@tc##1{\textcolor[rgb]{0.73,0.40,0.13}{##1}}}
\expandafter\def\csname PY@tok@sr\endcsname{\def\PY@tc##1{\textcolor[rgb]{0.73,0.40,0.53}{##1}}}
\expandafter\def\csname PY@tok@ss\endcsname{\def\PY@tc##1{\textcolor[rgb]{0.10,0.09,0.49}{##1}}}
\expandafter\def\csname PY@tok@sx\endcsname{\def\PY@tc##1{\textcolor[rgb]{0.00,0.50,0.00}{##1}}}
\expandafter\def\csname PY@tok@m\endcsname{\def\PY@tc##1{\textcolor[rgb]{0.40,0.40,0.40}{##1}}}
\expandafter\def\csname PY@tok@gh\endcsname{\let\PY@bf=\textbf\def\PY@tc##1{\textcolor[rgb]{0.00,0.00,0.50}{##1}}}
\expandafter\def\csname PY@tok@gu\endcsname{\let\PY@bf=\textbf\def\PY@tc##1{\textcolor[rgb]{0.50,0.00,0.50}{##1}}}
\expandafter\def\csname PY@tok@gd\endcsname{\def\PY@tc##1{\textcolor[rgb]{0.63,0.00,0.00}{##1}}}
\expandafter\def\csname PY@tok@gi\endcsname{\def\PY@tc##1{\textcolor[rgb]{0.00,0.63,0.00}{##1}}}
\expandafter\def\csname PY@tok@gr\endcsname{\def\PY@tc##1{\textcolor[rgb]{1.00,0.00,0.00}{##1}}}
\expandafter\def\csname PY@tok@ge\endcsname{\let\PY@it=\textit}
\expandafter\def\csname PY@tok@gs\endcsname{\let\PY@bf=\textbf}
\expandafter\def\csname PY@tok@gp\endcsname{\let\PY@bf=\textbf\def\PY@tc##1{\textcolor[rgb]{0.00,0.00,0.50}{##1}}}
\expandafter\def\csname PY@tok@go\endcsname{\def\PY@tc##1{\textcolor[rgb]{0.53,0.53,0.53}{##1}}}
\expandafter\def\csname PY@tok@gt\endcsname{\def\PY@tc##1{\textcolor[rgb]{0.00,0.27,0.87}{##1}}}
\expandafter\def\csname PY@tok@err\endcsname{\def\PY@bc##1{\setlength{\fboxsep}{0pt}\fcolorbox[rgb]{1.00,0.00,0.00}{1,1,1}{\strut ##1}}}
\expandafter\def\csname PY@tok@kc\endcsname{\let\PY@bf=\textbf\def\PY@tc##1{\textcolor[rgb]{0.00,0.50,0.00}{##1}}}
\expandafter\def\csname PY@tok@kd\endcsname{\let\PY@bf=\textbf\def\PY@tc##1{\textcolor[rgb]{0.00,0.50,0.00}{##1}}}
\expandafter\def\csname PY@tok@kn\endcsname{\let\PY@bf=\textbf\def\PY@tc##1{\textcolor[rgb]{0.00,0.50,0.00}{##1}}}
\expandafter\def\csname PY@tok@kr\endcsname{\let\PY@bf=\textbf\def\PY@tc##1{\textcolor[rgb]{0.00,0.50,0.00}{##1}}}
\expandafter\def\csname PY@tok@bp\endcsname{\def\PY@tc##1{\textcolor[rgb]{0.00,0.50,0.00}{##1}}}
\expandafter\def\csname PY@tok@fm\endcsname{\def\PY@tc##1{\textcolor[rgb]{0.00,0.00,1.00}{##1}}}
\expandafter\def\csname PY@tok@vc\endcsname{\def\PY@tc##1{\textcolor[rgb]{0.10,0.09,0.49}{##1}}}
\expandafter\def\csname PY@tok@vg\endcsname{\def\PY@tc##1{\textcolor[rgb]{0.10,0.09,0.49}{##1}}}
\expandafter\def\csname PY@tok@vi\endcsname{\def\PY@tc##1{\textcolor[rgb]{0.10,0.09,0.49}{##1}}}
\expandafter\def\csname PY@tok@vm\endcsname{\def\PY@tc##1{\textcolor[rgb]{0.10,0.09,0.49}{##1}}}
\expandafter\def\csname PY@tok@sa\endcsname{\def\PY@tc##1{\textcolor[rgb]{0.73,0.13,0.13}{##1}}}
\expandafter\def\csname PY@tok@sb\endcsname{\def\PY@tc##1{\textcolor[rgb]{0.73,0.13,0.13}{##1}}}
\expandafter\def\csname PY@tok@sc\endcsname{\def\PY@tc##1{\textcolor[rgb]{0.73,0.13,0.13}{##1}}}
\expandafter\def\csname PY@tok@dl\endcsname{\def\PY@tc##1{\textcolor[rgb]{0.73,0.13,0.13}{##1}}}
\expandafter\def\csname PY@tok@s2\endcsname{\def\PY@tc##1{\textcolor[rgb]{0.73,0.13,0.13}{##1}}}
\expandafter\def\csname PY@tok@sh\endcsname{\def\PY@tc##1{\textcolor[rgb]{0.73,0.13,0.13}{##1}}}
\expandafter\def\csname PY@tok@s1\endcsname{\def\PY@tc##1{\textcolor[rgb]{0.73,0.13,0.13}{##1}}}
\expandafter\def\csname PY@tok@mb\endcsname{\def\PY@tc##1{\textcolor[rgb]{0.40,0.40,0.40}{##1}}}
\expandafter\def\csname PY@tok@mf\endcsname{\def\PY@tc##1{\textcolor[rgb]{0.40,0.40,0.40}{##1}}}
\expandafter\def\csname PY@tok@mh\endcsname{\def\PY@tc##1{\textcolor[rgb]{0.40,0.40,0.40}{##1}}}
\expandafter\def\csname PY@tok@mi\endcsname{\def\PY@tc##1{\textcolor[rgb]{0.40,0.40,0.40}{##1}}}
\expandafter\def\csname PY@tok@il\endcsname{\def\PY@tc##1{\textcolor[rgb]{0.40,0.40,0.40}{##1}}}
\expandafter\def\csname PY@tok@mo\endcsname{\def\PY@tc##1{\textcolor[rgb]{0.40,0.40,0.40}{##1}}}
\expandafter\def\csname PY@tok@ch\endcsname{\let\PY@it=\textit\def\PY@tc##1{\textcolor[rgb]{0.25,0.50,0.50}{##1}}}
\expandafter\def\csname PY@tok@cm\endcsname{\let\PY@it=\textit\def\PY@tc##1{\textcolor[rgb]{0.25,0.50,0.50}{##1}}}
\expandafter\def\csname PY@tok@cpf\endcsname{\let\PY@it=\textit\def\PY@tc##1{\textcolor[rgb]{0.25,0.50,0.50}{##1}}}
\expandafter\def\csname PY@tok@c1\endcsname{\let\PY@it=\textit\def\PY@tc##1{\textcolor[rgb]{0.25,0.50,0.50}{##1}}}
\expandafter\def\csname PY@tok@cs\endcsname{\let\PY@it=\textit\def\PY@tc##1{\textcolor[rgb]{0.25,0.50,0.50}{##1}}}

\def\PYZbs{\char`\\}
\def\PYZus{\char`\_}
\def\PYZob{\char`\{}
\def\PYZcb{\char`\}}
\def\PYZca{\char`\^}
\def\PYZam{\char`\&}
\def\PYZlt{\char`\<}
\def\PYZgt{\char`\>}
\def\PYZsh{\char`\#}
\def\PYZpc{\char`\%}
\def\PYZdl{\char`\$}
\def\PYZhy{\char`\-}
\def\PYZsq{\char`\'}
\def\PYZdq{\char`\"}
\def\PYZti{\char`\~}
% for compatibility with earlier versions
\def\PYZat{@}
\def\PYZlb{[}
\def\PYZrb{]}
\makeatother


    % Exact colors from NB
    \definecolor{incolor}{rgb}{0.0, 0.0, 0.5}
    \definecolor{outcolor}{rgb}{0.545, 0.0, 0.0}



    
    % Prevent overflowing lines due to hard-to-break entities
    \sloppy 
    % Setup hyperref package
    \hypersetup{
      breaklinks=true,  % so long urls are correctly broken across lines
      colorlinks=true,
      urlcolor=urlcolor,
      linkcolor=linkcolor,
      citecolor=citecolor,
      }
    % Slightly bigger margins than the latex defaults
    
    \geometry{verbose,tmargin=1in,bmargin=1in,lmargin=1in,rmargin=1in}
    
    

    \begin{document}
    
    
    \maketitle
    
    

    
    \section{Manual strategy}\label{manual-strategy}

ML for trading Udacity Course exercise

More info: https://quantsoftware.gatech.edu/Manual\_strategy

A transcription of the Udacity Course lectures can be find on
https://docs.google.com/document/d/1ELqlnuTSdc9-MDHOkV0uvSY4RmI1eslyQlU9DgOY\_jc/edit?usp=sharing

Kairoart 2018 """

    \subsection{Overview}\label{overview}

In this project you will develop a trading strategy using your intuition
and Technical Analysis, and test it against a stock using your market
simulator. In a later project, you will use your same indicators but
with Machine Learning (instead of your intuition) to create a trading
strategy. We hope Machine Learning will do better than your intuition,
but who knows?

    \subsection{Data Details, Dates and
Rules}\label{data-details-dates-and-rules}

\begin{itemize}
\tightlist
\item
  For your report, trade only the symbol JPM. This will enable us to
  more easily compare results.
\item
  You may use data from other symbols (such as SPY) to inform your
  strategy.
\item
  The in sample/development period is January 1, 2008 to December 31
  2009.
\item
  The out of sample/testing period is January 1, 2010 to December 31
  2011.
\item
  Starting cash is 100,000.
\item
  Allowable positions are 1000 shares long, 1000 shares short, 0 shares.
\item
  Benchmark: The performance of a portfolio starting with 100,000 cash,
  investing in 1000 shares of JPM and holding that position.
\item
  There is no limit on leverage.
\item
  Transaction costs for ManualStrategy: Commission: 9.95, Impact: 0.005.
\item
  Transaction costs for TheoreticallyOptimalStrategy: Commission: 0.00,
  Impact: 0.00.
\end{itemize}

    \subsection{Task}\label{task}

Your code that implements your indicators as functions that operate on
dataframes. The code should generate the charts that illustrate your
indicators in the report.

\textbf{Indicators}

We'll use three tecnical indicators: * Bolliner Bands. * Simple Moving
Average (SMA) * Relative Strength Index (RSI)

    \subsection{Import libraries}\label{import-libraries}

    \begin{Verbatim}[commandchars=\\\{\}]
{\color{incolor}In [{\color{incolor}1}]:} \PY{k+kn}{import} \PY{n+nn}{pandas} \PY{k}{as} \PY{n+nn}{pd}
        \PY{k+kn}{import} \PY{n+nn}{numpy} \PY{k}{as} \PY{n+nn}{np}
        \PY{k+kn}{import} \PY{n+nn}{datetime} \PY{k}{as} \PY{n+nn}{dt}
        \PY{k+kn}{import} \PY{n+nn}{math}
        \PY{k+kn}{import} \PY{n+nn}{matplotlib}\PY{n+nn}{.}\PY{n+nn}{pyplot} \PY{k}{as} \PY{n+nn}{plt}
        \PY{o}{\PYZpc{}}\PY{k}{matplotlib} inline
        
        
        \PY{c+c1}{\PYZsh{} Add parent directory PATH for looking for modules,}
        \PY{k+kn}{import} \PY{n+nn}{sys}
        \PY{c+c1}{\PYZsh{}sys.path.insert(0,\PYZsq{}..\PYZsq{})}
        \PY{k+kn}{from} \PY{n+nn}{util} \PY{k}{import} \PY{n}{get\PYZus{}data}\PY{p}{,} \PY{n}{plot\PYZus{}data}
        
        \PY{c+c1}{\PYZsh{} Indicators}
        \PY{k+kn}{import} \PY{n+nn}{indicators}
        \PY{k+kn}{from} \PY{n+nn}{indicators} \PY{k}{import} \PY{n}{get\PYZus{}momentum}\PY{p}{,} \PY{n}{get\PYZus{}rolling\PYZus{}mean}\PY{p}{,} \PY{n}{get\PYZus{}rolling\PYZus{}std}\PY{p}{,} \PY{n}{get\PYZus{}bollinger\PYZus{}bands}\PY{p}{,} \PY{n}{get\PYZus{}sma}\PY{p}{,} \PY{n}{get\PYZus{}RSI}\PY{p}{,} \PY{n}{plot\PYZus{}bollinger}\PY{p}{,}  \PY{n}{plot\PYZus{}momentum}\PY{p}{,} \PY{n}{plot\PYZus{}sma\PYZus{}indicator}\PY{p}{,} \PY{n}{plot\PYZus{}rsi\PYZus{}indicator}
\end{Verbatim}


    
    
    \subsection{Part 1: Technical
Indicators}\label{part-1-technical-indicators}

Develop and describe at least 3 and at most 5 technical indicators. You
may find our lecture on time series processing to be helpful. For each
indicator you should create a single, compelling chart that illustrates
the indicator.

As an example, you might create a chart that shows the price history of
the stock, along with "helper data" (such as upper and lower bollinger
bands) and the value of the indicator itself. Another example: If you
were using price/SMA as an indicator you would want to create a chart
with 3 lines: Price, SMA, Price/SMA. In order to facilitate
visualization of the indicator you might normalize the data to 1.0 at
the start of the date range (i.e. divide price{[}t{]} by price{[}0{]}).

Your report description of each indicator should enable someone to
reproduce it just by reading the description. We want a written
description here, not code, however, it is OK to augment your written
description with a pseudocode figure.

At least one of the indicators you use should be completely different
from the ones presented in our lectures. (i.e. something other than SMA,
Bollinger Bands, RSI).

    \subsubsection{Momentum}\label{momentum}

    \begin{Verbatim}[commandchars=\\\{\}]
{\color{incolor}In [{\color{incolor}2}]:} \PY{l+s+sd}{\PYZdq{}\PYZdq{}\PYZdq{}Momentum.\PYZdq{}\PYZdq{}\PYZdq{}}
        
        \PY{k}{def} \PY{n+nf}{test\PYZus{}run}\PY{p}{(}\PY{n}{plotly}\PY{o}{=}\PY{k+kc}{False}\PY{p}{)}\PY{p}{:}
            \PY{c+c1}{\PYZsh{} Read data}
            \PY{n}{dates} \PY{o}{=} \PY{n}{pd}\PY{o}{.}\PY{n}{date\PYZus{}range}\PY{p}{(}\PY{l+s+s1}{\PYZsq{}}\PY{l+s+s1}{2008\PYZhy{}01\PYZhy{}01}\PY{l+s+s1}{\PYZsq{}}\PY{p}{,} \PY{l+s+s1}{\PYZsq{}}\PY{l+s+s1}{2009\PYZhy{}12\PYZhy{}31}\PY{l+s+s1}{\PYZsq{}}\PY{p}{)}
            \PY{n}{symbols} \PY{o}{=} \PY{p}{[}\PY{l+s+s1}{\PYZsq{}}\PY{l+s+s1}{JPM}\PY{l+s+s1}{\PYZsq{}}\PY{p}{]}
            \PY{n}{df} \PY{o}{=} \PY{n}{get\PYZus{}data}\PY{p}{(}\PY{n}{symbols}\PY{p}{,} \PY{n}{dates}\PY{p}{)}
        
            \PY{c+c1}{\PYZsh{} 1. Normalize the prices Dataframe}
            \PY{n}{normed} \PY{o}{=} \PY{n}{pd}\PY{o}{.}\PY{n}{DataFrame}\PY{p}{(}\PY{p}{)}
            \PY{k}{for} \PY{n}{column} \PY{o+ow}{in} \PY{n}{df}\PY{p}{:}
                \PY{n}{normed}\PY{p}{[}\PY{n}{column}\PY{p}{]} \PY{o}{=} \PY{n}{df}\PY{p}{[}\PY{n}{column}\PY{p}{]}\PY{o}{.}\PY{n}{values} \PY{o}{/} \PY{n}{df}\PY{p}{[}\PY{n}{column}\PY{p}{]}\PY{o}{.}\PY{n}{iloc}\PY{p}{[}\PY{l+m+mi}{0}\PY{p}{]}\PY{p}{;}
            
            \PY{c+c1}{\PYZsh{} 2. Compute momentum}
            \PY{n}{sym\PYZus{}mom} \PY{o}{=} \PY{n}{get\PYZus{}momentum}\PY{p}{(}\PY{n}{normed}\PY{p}{[}\PY{n}{column}\PY{p}{]}\PY{p}{,} \PY{n}{window}\PY{o}{=}\PY{l+m+mi}{5}\PY{p}{)}
        
            \PY{c+c1}{\PYZsh{} 3. Plot raw JPM values and Momentym}
            \PY{n}{plot\PYZus{}momentum}\PY{p}{(}\PY{n}{dates}\PY{p}{,} \PY{n}{df}\PY{o}{.}\PY{n}{index}\PY{p}{,} \PY{n}{normed}\PY{p}{[}\PY{n}{column}\PY{p}{]}\PY{p}{,} \PY{n}{sym\PYZus{}mom}\PY{p}{,} \PY{n}{title}\PY{o}{=}\PY{l+s+s2}{\PYZdq{}}\PY{l+s+s2}{Momentum Indicator}\PY{l+s+s2}{\PYZdq{}}\PY{p}{,}
                          \PY{n}{fig\PYZus{}size}\PY{o}{=}\PY{p}{(}\PY{l+m+mi}{12}\PY{p}{,} \PY{l+m+mi}{6}\PY{p}{)}\PY{p}{)}
            
            \PY{c+c1}{\PYZsh{}plot\PYZus{}momentum(normed[column], sym\PYZus{}mom, title=\PYZdq{}Momentum Indicator\PYZdq{},}
            \PY{c+c1}{\PYZsh{}              fig\PYZus{}size=(12, 6))}
\end{Verbatim}


    \begin{Verbatim}[commandchars=\\\{\}]
{\color{incolor}In [{\color{incolor}3}]:} \PY{n}{test\PYZus{}run}\PY{p}{(}\PY{p}{)}
\end{Verbatim}


    \begin{center}
    \adjustimage{max size={0.9\linewidth}{0.9\paperheight}}{output_9_0.png}
    \end{center}
    { \hspace*{\fill} \\}
    
    \paragraph{Description}\label{description}

\textbf{Type} It's a trend indicator.

\textbf{Momentum}

\begin{itemize}
\tightlist
\item
  It refers to the rate of change on price movements for a particular
  asset -- that is, the speed at which the price is changing.
\item
  If a trader wants to use a momentum-based strategy, he takes a long
  position in a stock or asset that has been trending up. If the stock
  is trending down, he takes a short position. Instead of the
  traditional philosophy of trading -- buy low, sell high -- momentum
  investing seeks to sell low and buy lower, or buy high and sell
  higher.
\end{itemize}

\textbf{Buy signal}

If the trend line is up, the momentum investor buys the stock.

\textbf{Sell signal}

If the trend line is down, the momentum investor sells the stock.

    \subsubsection{Bollinger Bands}\label{bollinger-bands}

    \begin{Verbatim}[commandchars=\\\{\}]
{\color{incolor}In [{\color{incolor}4}]:} \PY{l+s+sd}{\PYZdq{}\PYZdq{}\PYZdq{}Bollinger Bands.\PYZdq{}\PYZdq{}\PYZdq{}}
        
        \PY{k}{def} \PY{n+nf}{test\PYZus{}run}\PY{p}{(}\PY{n}{plotly}\PY{o}{=}\PY{k+kc}{False}\PY{p}{)}\PY{p}{:}
            \PY{c+c1}{\PYZsh{} Read data}
            \PY{n}{dates} \PY{o}{=} \PY{n}{pd}\PY{o}{.}\PY{n}{date\PYZus{}range}\PY{p}{(}\PY{l+s+s1}{\PYZsq{}}\PY{l+s+s1}{2008\PYZhy{}01\PYZhy{}01}\PY{l+s+s1}{\PYZsq{}}\PY{p}{,} \PY{l+s+s1}{\PYZsq{}}\PY{l+s+s1}{2009\PYZhy{}12\PYZhy{}31}\PY{l+s+s1}{\PYZsq{}}\PY{p}{)}
            \PY{n}{symbols} \PY{o}{=} \PY{p}{[}\PY{l+s+s1}{\PYZsq{}}\PY{l+s+s1}{JPM}\PY{l+s+s1}{\PYZsq{}}\PY{p}{]}
            \PY{n}{df} \PY{o}{=} \PY{n}{get\PYZus{}data}\PY{p}{(}\PY{n}{symbols}\PY{p}{,} \PY{n}{dates}\PY{p}{)}
        
            \PY{c+c1}{\PYZsh{} Compute Bollinger Bands}
            \PY{c+c1}{\PYZsh{} 1. Compute rolling mean}
            \PY{n}{rm\PYZus{}JPM} \PY{o}{=} \PY{n}{get\PYZus{}rolling\PYZus{}mean}\PY{p}{(}\PY{n}{df}\PY{p}{[}\PY{l+s+s1}{\PYZsq{}}\PY{l+s+s1}{JPM}\PY{l+s+s1}{\PYZsq{}}\PY{p}{]}\PY{p}{,} \PY{n}{window}\PY{o}{=}\PY{l+m+mi}{10}\PY{p}{)}
        
            \PY{c+c1}{\PYZsh{} 2. Compute rolling standard deviation}
            \PY{n}{rstd\PYZus{}JPM} \PY{o}{=} \PY{n}{get\PYZus{}rolling\PYZus{}std}\PY{p}{(}\PY{n}{df}\PY{p}{[}\PY{l+s+s1}{\PYZsq{}}\PY{l+s+s1}{JPM}\PY{l+s+s1}{\PYZsq{}}\PY{p}{]}\PY{p}{,} \PY{n}{window}\PY{o}{=}\PY{l+m+mi}{10}\PY{p}{)}
        
            \PY{c+c1}{\PYZsh{} 3. Compute upper and lower bands}
            \PY{n}{upper\PYZus{}band}\PY{p}{,} \PY{n}{lower\PYZus{}band} \PY{o}{=} \PY{n}{get\PYZus{}bollinger\PYZus{}bands}\PY{p}{(}\PY{n}{rm\PYZus{}JPM}\PY{p}{,} \PY{n}{rstd\PYZus{}JPM}\PY{p}{)}
            
            \PY{c+c1}{\PYZsh{} 4. Plot raw JPM values, rolling mean and Bollinger Bands}
            \PY{n}{plot\PYZus{}bollinger}\PY{p}{(}\PY{n}{dates}\PY{p}{,} \PY{n}{df}\PY{o}{.}\PY{n}{index}\PY{p}{,} \PY{n}{df}\PY{p}{[}\PY{l+s+s1}{\PYZsq{}}\PY{l+s+s1}{JPM}\PY{l+s+s1}{\PYZsq{}}\PY{p}{]}\PY{p}{,} \PY{n}{upper\PYZus{}band}\PY{p}{,} \PY{n}{lower\PYZus{}band}\PY{p}{,} \PY{n}{rm\PYZus{}JPM}\PY{p}{,} 
                           \PY{n}{num\PYZus{}std}\PY{o}{=}\PY{l+m+mi}{1}\PY{p}{,} \PY{n}{title}\PY{o}{=}\PY{l+s+s2}{\PYZdq{}}\PY{l+s+s2}{Bollinger Indicator}\PY{l+s+s2}{\PYZdq{}}\PY{p}{,} \PY{n}{fig\PYZus{}size}\PY{o}{=}\PY{p}{(}\PY{l+m+mi}{12}\PY{p}{,} \PY{l+m+mi}{6}\PY{p}{)}\PY{p}{)}
\end{Verbatim}


    \begin{Verbatim}[commandchars=\\\{\}]
{\color{incolor}In [{\color{incolor}5}]:} \PY{n}{test\PYZus{}run}\PY{p}{(}\PY{p}{)}
\end{Verbatim}


    \begin{center}
    \adjustimage{max size={0.9\linewidth}{0.9\paperheight}}{output_13_0.png}
    \end{center}
    { \hspace*{\fill} \\}
    
    \subsubsection{Simple moving average
(SMA)}\label{simple-moving-average-sma}

    \begin{Verbatim}[commandchars=\\\{\}]
{\color{incolor}In [{\color{incolor}6}]:} \PY{l+s+sd}{\PYZdq{}\PYZdq{}\PYZdq{}Simple moving average (SMA)\PYZdq{}\PYZdq{}\PYZdq{}}
        
        \PY{k}{def} \PY{n+nf}{test\PYZus{}run}\PY{p}{(}\PY{n}{plotly}\PY{o}{=}\PY{k+kc}{False}\PY{p}{)}\PY{p}{:}
            \PY{c+c1}{\PYZsh{} Read data}
            \PY{n}{dates} \PY{o}{=} \PY{n}{pd}\PY{o}{.}\PY{n}{date\PYZus{}range}\PY{p}{(}\PY{l+s+s1}{\PYZsq{}}\PY{l+s+s1}{2008\PYZhy{}01\PYZhy{}01}\PY{l+s+s1}{\PYZsq{}}\PY{p}{,} \PY{l+s+s1}{\PYZsq{}}\PY{l+s+s1}{2009\PYZhy{}12\PYZhy{}31}\PY{l+s+s1}{\PYZsq{}}\PY{p}{)}
            \PY{n}{symbols} \PY{o}{=} \PY{p}{[}\PY{l+s+s1}{\PYZsq{}}\PY{l+s+s1}{JPM}\PY{l+s+s1}{\PYZsq{}}\PY{p}{]}
            \PY{n}{df} \PY{o}{=} \PY{n}{get\PYZus{}data}\PY{p}{(}\PY{n}{symbols}\PY{p}{,} \PY{n}{dates}\PY{p}{)}
        
            \PY{c+c1}{\PYZsh{} 1. Normalize the prices Dataframe}
            \PY{n}{normed} \PY{o}{=} \PY{n}{pd}\PY{o}{.}\PY{n}{DataFrame}\PY{p}{(}\PY{p}{)}
            \PY{k}{for} \PY{n}{column} \PY{o+ow}{in} \PY{n}{df}\PY{p}{:}
                \PY{n}{normed}\PY{p}{[}\PY{n}{column}\PY{p}{]} \PY{o}{=} \PY{n}{df}\PY{p}{[}\PY{n}{column}\PY{p}{]}\PY{o}{.}\PY{n}{values} \PY{o}{/} \PY{n}{df}\PY{p}{[}\PY{n}{column}\PY{p}{]}\PY{o}{.}\PY{n}{iloc}\PY{p}{[}\PY{l+m+mi}{0}\PY{p}{]}\PY{p}{;}
            
            \PY{c+c1}{\PYZsh{} 2. Compute SMA}
            \PY{n}{sma\PYZus{}JPM}\PY{p}{,} \PY{n}{q} \PY{o}{=} \PY{n}{get\PYZus{}sma}\PY{p}{(}\PY{n}{normed}\PY{p}{[}\PY{n}{column}\PY{p}{]}\PY{p}{,} \PY{n}{window}\PY{o}{=}\PY{l+m+mi}{10}\PY{p}{)}
        
            \PY{c+c1}{\PYZsh{} 3. Plot symbol values, SMA and SMA quality}
            \PY{n}{plot\PYZus{}sma\PYZus{}indicator}\PY{p}{(}\PY{n}{dates}\PY{p}{,} \PY{n}{df}\PY{o}{.}\PY{n}{index}\PY{p}{,} \PY{n}{normed}\PY{p}{[}\PY{n}{column}\PY{p}{]}\PY{p}{,} \PY{n}{sma\PYZus{}JPM}\PY{p}{,} \PY{n}{q}\PY{p}{,} \PY{l+s+s2}{\PYZdq{}}\PY{l+s+s2}{Simple Moving Average (SMA)}\PY{l+s+s2}{\PYZdq{}}\PY{p}{)}
                
            
\end{Verbatim}


    \begin{Verbatim}[commandchars=\\\{\}]
{\color{incolor}In [{\color{incolor}7}]:} \PY{n}{test\PYZus{}run}\PY{p}{(}\PY{k+kc}{False}\PY{p}{)}
\end{Verbatim}


    \begin{center}
    \adjustimage{max size={0.9\linewidth}{0.9\paperheight}}{output_16_0.png}
    \end{center}
    { \hspace*{\fill} \\}
    
    \paragraph{Description}\label{description}

The moving average is one of the most widely used technical indicators.
It is used along with other technical indicators or it can form the
building block for the computation of other technical indicators.

A ``moving average'' is average of the asset prices over the ``x''
number of days/weeks. The term ``moving'' is used because the group of
data moves forward with each new trading day. For each new day, we
include the price of that day and exclude the price of the first day in
the data sequence.

\textbf{Type} It's a trend indicator.

\textbf{Estimation}

To compute a 10-day window SMA, we take the sum of prices over 10 days
and divide it by 10. To arrive at the next data point for the 20-day
SMA, we include the price of the next trading day while excluding the
price of the first trading day. This way the group of data moves
forward.

The SMA assigns equal weights to each price point in the group.

\textbf{Analysis}

The moving average tells whether a trend has begun, ended or reversed.
The averaging of the prices produces a smoother line which makes it
easier to identify the underlying trend. However, the moving average
lags the market action.

A shorter moving average is more sensitive than a longer moving average.
However, it is prone to generate false trading signals.

\textbf{Signals}

When the \emph{closing price moves above the moving average, a buy
signal is generated and vice versa}. When using a single moving average
one should select the averaging period in such a way that it is
sensitive in generating trading signals and at the same time insensitive
in giving out false signals.

\textbf{Quantifying}

We lso need a way to quantify it, to turn it into a number. The way we
do that is to compare the current price with the current simple moving
average, and construct a ratio. So, if, for instance, the price is 10\%
above the simple moving average, we'd end up getting a positive 0.1. If
it were 10\% below, we'd get a negative 0.1.

    \subsubsection{Relative Strength Index
(RSI)}\label{relative-strength-index-rsi}

    \begin{Verbatim}[commandchars=\\\{\}]
{\color{incolor}In [{\color{incolor}8}]:}    
        \PY{k}{def} \PY{n+nf}{test\PYZus{}run}\PY{p}{(}\PY{p}{)}\PY{p}{:}
            \PY{c+c1}{\PYZsh{} Read data}
            \PY{n}{dates} \PY{o}{=} \PY{n}{pd}\PY{o}{.}\PY{n}{date\PYZus{}range}\PY{p}{(}\PY{l+s+s1}{\PYZsq{}}\PY{l+s+s1}{2008\PYZhy{}01\PYZhy{}01}\PY{l+s+s1}{\PYZsq{}}\PY{p}{,} \PY{l+s+s1}{\PYZsq{}}\PY{l+s+s1}{2009\PYZhy{}12\PYZhy{}31}\PY{l+s+s1}{\PYZsq{}}\PY{p}{)}
            \PY{n}{symbols} \PY{o}{=} \PY{p}{[}\PY{l+s+s1}{\PYZsq{}}\PY{l+s+s1}{JPM}\PY{l+s+s1}{\PYZsq{}}\PY{p}{]}
            \PY{n}{df} \PY{o}{=} \PY{n}{get\PYZus{}data}\PY{p}{(}\PY{n}{symbols}\PY{p}{,} \PY{n}{dates}\PY{p}{)}
        
            \PY{c+c1}{\PYZsh{} 1. Compute RSI}
            \PY{n}{rsi\PYZus{}JPM} \PY{o}{=} \PY{n}{get\PYZus{}RSI}\PY{p}{(}\PY{n}{df}\PY{p}{[}\PY{l+s+s1}{\PYZsq{}}\PY{l+s+s1}{JPM}\PY{l+s+s1}{\PYZsq{}}\PY{p}{]}\PY{p}{)}
            
            \PY{c+c1}{\PYZsh{} 2. Plot RSI}
            \PY{n}{plot\PYZus{}rsi\PYZus{}indicator}\PY{p}{(}\PY{n}{dates}\PY{p}{,} \PY{n}{df}\PY{o}{.}\PY{n}{index}\PY{p}{,} \PY{n}{df}\PY{p}{[}\PY{l+s+s1}{\PYZsq{}}\PY{l+s+s1}{JPM}\PY{l+s+s1}{\PYZsq{}}\PY{p}{]}\PY{p}{,} \PY{n}{rsi\PYZus{}JPM}\PY{p}{,} \PY{n}{window}\PY{o}{=}\PY{l+m+mi}{14}\PY{p}{,} 
                               \PY{n}{title}\PY{o}{=}\PY{l+s+s2}{\PYZdq{}}\PY{l+s+s2}{RSI Indicator}\PY{l+s+s2}{\PYZdq{}}\PY{p}{,} \PY{n}{fig\PYZus{}size}\PY{o}{=}\PY{p}{(}\PY{l+m+mi}{12}\PY{p}{,} \PY{l+m+mi}{6}\PY{p}{)}\PY{p}{)}
\end{Verbatim}


    \begin{Verbatim}[commandchars=\\\{\}]
{\color{incolor}In [{\color{incolor}9}]:} \PY{n}{test\PYZus{}run}\PY{p}{(}\PY{p}{)}
\end{Verbatim}


    \begin{Verbatim}[commandchars=\\\{\}]
/home/emi/Jupyter/ML4T\_2018Spring/manual\_strategy/indicators.py:100: FutureWarning:

pd.rolling\_apply is deprecated for Series and will be removed in a future version, replace with 
	Series.rolling(window=14,center=False).apply(func=<function>,args=<tuple>,kwargs=<dict>)


    \end{Verbatim}

    \begin{center}
    \adjustimage{max size={0.9\linewidth}{0.9\paperheight}}{output_20_1.png}
    \end{center}
    { \hspace*{\fill} \\}
    
    \paragraph{Description}\label{description}

The Relative Strength Index - RSI is a momentum indicator that measures
the magnitude of recent price changes to analyze
\href{https://www.investopedia.com/terms/o/overbought.asp}{overbought}
or \href{https://www.investopedia.com/terms/o/oversold.asp}{oversold}
conditions. It is primarily used to attempt to identify overbought or
oversold conditions in the trading of an asset.

\textbf{Type} It's a momentum indicator.

\textbf{Estimation} The relative strength index (RSI) is calculated
using the following formula:

RSI = 100 - 100 / (1 + RS)

Where RS = Average gain of up periods during the specified time frame /
Average loss of down periods during the specified time frame

The RSI provides a relative evaluation of the strength of a security's
recent price performance, thus making it a momentum indicator. RSI
values range from 0 to 100. The default time frame for comparing up
periods to down periods is 14, as in 14 trading days.

\textbf{Analisys} Taking the prior value plus the current value is a
smoothing technique similar to that used in calculating an exponential
moving average. This also means that RSI values become more accurate as
the calculation period extends. To exactly replicate our RSI numbers, a
formula will need at least 250 data points.

The formula normalizes RS and turns it into an oscillator that
fluctuates between zero and 100. In fact, a plot of RS looks exactly the
same as a plot of RSI. The normalization step makes it easier to
identify extremes because RSI is range bound. RSI is 0 when the Average
Gain equals zero. Assuming a 14-period RSI, a zero RSI value means
prices moved lower all 14 periods. There were no gains to measure. RSI
is 100 when the Average Loss equals zero. This means prices moved higher
all 14 periods. There were no losses to measure.

\textbf{Signals}

\begin{itemize}
\tightlist
\item
  \textbf{Buy signal} occurs when the RSI moves below 50 and then back
  above it.
\item
  \textbf{Sell signal} occurs when the RSI moves above 50 and then back
  below it.
\item
  RSI values of 70 or above indicate that a security is becoming
  overbought or overvalued, and therefore, may be primed for a trend
  reversal or corrective
  {[}pullback{]}(https://www.investopedia.com/terms/p/pullback.asp in
  price.
\item
  RSI reading of 30 or below is commonly interpreted as indicating an
  oversold or undervalued condition that may signal a trend change or
  corrective price reversal to the upside.
\end{itemize}

    \section{Part 2: Theoretically Optimal
Strategy}\label{part-2-theoretically-optimal-strategy}

Assume that you can see the future, but that you are constrained by the
portfolio size and order limits as specified above. Create a set of
trades that represents the best a strategy could possibly do during the
\textbf{in sample period}. The reason we're having you do this is so
that you will have an idea of an upper bound on performance.

The intent is for you to use adjusted close prices with the market
simulator that you wrote earlier in the course. For this activity, use
0.00, and 0.0 for commissions and impact respectively.

\textbf{Main functions}

\begin{itemize}
\tightlist
\item
  \emph{Indicators:} Your code that implements your indicators as
  functions that operate on dataframes. The "main" code in indicators
  should generate the charts that illustrate your indicators in the
  report.
\item
  \emph{Marketsimcode:} An improved version of your marketsim code that
  accepts a "trades" data frame (instead of a file). More info on the
  trades data frame below.
\item
  \emph{ManualStrategy:} Code implementing a ManualStrategy object (your
  manual strategy). It should implement testPolicy() which returns a
  trades data frame (see below). The main part of this code should call
  marketsimcode as necessary to generate the plots used in the report.
\item
  \emph{TheoreticallyOptimalStrategy:} Code implementing a
  TheoreticallyOptimalStrategy object (details below). It should
  implement testPolicy() which returns a trades data frame (see below).
  The main part of this code should call marketsimcode as necessary to
  generate the plots used in the report.
\end{itemize}

Provide a chart that reports:

\begin{verbatim}
* Benchmark (see definition above) normalized to 1.0 at the start: Blue line
* Value of the theoretically optimal portfolio (normalized to 1.0 at the start): Black line
\end{verbatim}

You should also report in text:

\begin{verbatim}
* Cumulative return of the benchmark and portfolio
* Stdev of daily returns of benchmark and portfolio
* Mean of daily returns of benchmark and portfolio
\end{verbatim}

Your code should implement testPolicy() as follows:

df\_trades = tos.testPolicy(symbol = "AAPL", sd=dt.datetime(2010,1,1),
ed=dt.datetime(2011,12,31), sv = 100000)

\textbf{Input parameters}

\begin{verbatim}
* symbol: the stock symbol to act on
* sd: A datetime object that represents the start date
* ed: A datetime object that represents the end date
* sv: Start value of the portfolio
\end{verbatim}

\textbf{Output result}

df\_trades: A data frame whose values represent trades for each day.
Legal values are +1000.0 indicating a BUY of 1000 shares, -1000.0
indicating a SELL of 1000 shares, and 0.0 indicating NOTHING. Values of
+2000 and -2000 for trades are also legal so long as net holdings are
constrained to -1000, 0, and 1000.

    \subsection{Import libraries}\label{import-libraries}

    \begin{Verbatim}[commandchars=\\\{\}]
{\color{incolor}In [{\color{incolor}10}]:} \PY{c+c1}{\PYZsh{} For Theoritically Optimal Strategy}
         \PY{k+kn}{from} \PY{n+nn}{marketsim} \PY{k}{import} \PY{n}{market\PYZus{}simulator}
         \PY{k+kn}{from} \PY{n+nn}{best\PYZus{}strategy} \PY{k}{import} \PY{n}{BestPossibleStrategy}
\end{Verbatim}


    
    
    \subsection{Starting cash and symbol of
interest}\label{starting-cash-and-symbol-of-interest}

    \begin{Verbatim}[commandchars=\\\{\}]
{\color{incolor}In [{\color{incolor}11}]:} \PY{c+c1}{\PYZsh{} Specify the start and end dates for this period.}
         \PY{n}{start\PYZus{}val} \PY{o}{=} \PY{l+m+mi}{100000}
         \PY{n}{symbol} \PY{o}{=} \PY{l+s+s2}{\PYZdq{}}\PY{l+s+s2}{JPM}\PY{l+s+s2}{\PYZdq{}}
\end{Verbatim}


    \subsection{In-sample or training period
performance}\label{in-sample-or-training-period-performance}

Create a set of trades that represents the best a strategy could
possibly do during the \textbf{in sample period}

Chart: * Benchmark (see definition above) normalized to 1.0 at the
start: Blue line * Value of the theoretically optimal portfolio
(normalized to 1.0 at the start): Black line

Report: * Cumulative return of the benchmark and portfolio * Stdev of
daily returns of benchmark and portfolio * Mean of daily returns of
benchmark and portfolio

    \begin{Verbatim}[commandchars=\\\{\}]
{\color{incolor}In [{\color{incolor}21}]:} \PY{c+c1}{\PYZsh{} Start and end dates for this period}
         \PY{n}{start\PYZus{}d} \PY{o}{=} \PY{n}{dt}\PY{o}{.}\PY{n}{datetime}\PY{p}{(}\PY{l+m+mi}{2008}\PY{p}{,} \PY{l+m+mi}{1}\PY{p}{,} \PY{l+m+mi}{1}\PY{p}{)}
         \PY{n}{end\PYZus{}d} \PY{o}{=} \PY{n}{dt}\PY{o}{.}\PY{n}{datetime}\PY{p}{(}\PY{l+m+mi}{2009}\PY{p}{,} \PY{l+m+mi}{12}\PY{p}{,} \PY{l+m+mi}{31}\PY{p}{)}
         
         \PY{c+c1}{\PYZsh{} Get benchmark data}
         \PY{n}{benchmark\PYZus{}prices} \PY{o}{=} \PY{n}{get\PYZus{}data}\PY{p}{(}\PY{p}{[}\PY{n}{symbol}\PY{p}{]}\PY{p}{,} \PY{n}{pd}\PY{o}{.}\PY{n}{date\PYZus{}range}\PY{p}{(}\PY{n}{start\PYZus{}d}\PY{p}{,} \PY{n}{end\PYZus{}d}\PY{p}{)}\PY{p}{,} 
             \PY{n}{addSPY}\PY{o}{=}\PY{k+kc}{False}\PY{p}{)}\PY{o}{.}\PY{n}{dropna}\PY{p}{(}\PY{p}{)}
         
         \PY{c+c1}{\PYZsh{} Create benchmark trades: buy 1000 shares of symbol, hold them till the last date}
         \PY{n}{df\PYZus{}benchmark\PYZus{}trades} \PY{o}{=} \PY{n}{pd}\PY{o}{.}\PY{n}{DataFrame}\PY{p}{(}
             \PY{n}{data}\PY{o}{=}\PY{p}{[}\PY{p}{(}\PY{n}{benchmark\PYZus{}prices}\PY{o}{.}\PY{n}{index}\PY{o}{.}\PY{n}{min}\PY{p}{(}\PY{p}{)}\PY{p}{,} \PY{n}{symbol}\PY{p}{,} \PY{l+s+s2}{\PYZdq{}}\PY{l+s+s2}{BUY}\PY{l+s+s2}{\PYZdq{}}\PY{p}{,} \PY{l+m+mi}{1000}\PY{p}{)}\PY{p}{,} 
             \PY{p}{(}\PY{n}{benchmark\PYZus{}prices}\PY{o}{.}\PY{n}{index}\PY{o}{.}\PY{n}{max}\PY{p}{(}\PY{p}{)}\PY{p}{,} \PY{n}{symbol}\PY{p}{,} \PY{l+s+s2}{\PYZdq{}}\PY{l+s+s2}{SELL}\PY{l+s+s2}{\PYZdq{}}\PY{p}{,} \PY{l+m+mi}{1000}\PY{p}{)}\PY{p}{]}\PY{p}{,} 
             \PY{n}{columns}\PY{o}{=}\PY{p}{[}\PY{l+s+s2}{\PYZdq{}}\PY{l+s+s2}{Date}\PY{l+s+s2}{\PYZdq{}}\PY{p}{,} \PY{l+s+s2}{\PYZdq{}}\PY{l+s+s2}{Symbol}\PY{l+s+s2}{\PYZdq{}}\PY{p}{,} \PY{l+s+s2}{\PYZdq{}}\PY{l+s+s2}{Order}\PY{l+s+s2}{\PYZdq{}}\PY{p}{,} \PY{l+s+s2}{\PYZdq{}}\PY{l+s+s2}{Shares}\PY{l+s+s2}{\PYZdq{}}\PY{p}{]}\PY{p}{)}
         \PY{n}{df\PYZus{}benchmark\PYZus{}trades}\PY{o}{.}\PY{n}{set\PYZus{}index}\PY{p}{(}\PY{l+s+s2}{\PYZdq{}}\PY{l+s+s2}{Date}\PY{l+s+s2}{\PYZdq{}}\PY{p}{,} \PY{n}{inplace}\PY{o}{=}\PY{k+kc}{True}\PY{p}{)}
         
         \PY{c+c1}{\PYZsh{} Create an instance of BestPossibleStrategy}
         \PY{n}{best\PYZus{}poss} \PY{o}{=} \PY{n}{BestPossibleStrategy}\PY{p}{(}\PY{p}{)}
         \PY{n}{df\PYZus{}trades} \PY{o}{=} \PY{n}{best\PYZus{}poss}\PY{o}{.}\PY{n}{test\PYZus{}policy}\PY{p}{(}\PY{n}{symbol}\PY{o}{=}\PY{n}{symbol}\PY{p}{,} \PY{n}{start\PYZus{}date}\PY{o}{=}\PY{n}{start\PYZus{}d}\PY{p}{,} \PY{n}{end\PYZus{}date}\PY{o}{=}\PY{n}{end\PYZus{}d}\PY{p}{)}
         
         \PY{c+c1}{\PYZsh{} Retrieve performance stats via a market simulator}
         \PY{n+nb}{print} \PY{p}{(}\PY{l+s+s2}{\PYZdq{}}\PY{l+s+s2}{Performances during training period for }\PY{l+s+si}{\PYZob{}\PYZcb{}}\PY{l+s+s2}{\PYZdq{}}\PY{o}{.}\PY{n}{format}\PY{p}{(}\PY{n}{symbol}\PY{p}{)}\PY{p}{)}
         \PY{n+nb}{print} \PY{p}{(}\PY{l+s+s2}{\PYZdq{}}\PY{l+s+s2}{Date Range: }\PY{l+s+si}{\PYZob{}\PYZcb{}}\PY{l+s+s2}{ to }\PY{l+s+si}{\PYZob{}\PYZcb{}}\PY{l+s+s2}{\PYZdq{}}\PY{o}{.}\PY{n}{format}\PY{p}{(}\PY{n}{start\PYZus{}d}\PY{p}{,} \PY{n}{end\PYZus{}d}\PY{p}{)}\PY{p}{)}
         \PY{n}{orders\PYZus{}count\PYZus{}out}\PY{p}{,} \PY{n}{sharpe\PYZus{}ratio\PYZus{}out}\PY{p}{,} \PY{n}{cum\PYZus{}ret\PYZus{}out}\PY{p}{,} \PY{n}{std\PYZus{}daily\PYZus{}ret\PYZus{}out}\PY{p}{,} \PY{n}{avg\PYZus{}daily\PYZus{}ret\PYZus{}out}\PY{p}{,} \PY{n}{final\PYZus{}value\PYZus{}out} \PY{o}{=} \PY{n}{market\PYZus{}simulator}\PY{p}{(}\PY{n}{df\PYZus{}trades}\PY{p}{,} \PY{n}{df\PYZus{}benchmark\PYZus{}trades}\PY{p}{,} \PY{n}{start\PYZus{}val}\PY{o}{=}\PY{n}{start\PYZus{}val}\PY{p}{,} \PY{n}{commission}\PY{o}{=}\PY{l+m+mi}{0}\PY{p}{,} 
             \PY{n}{impact}\PY{o}{=}\PY{l+m+mi}{0}\PY{p}{)}
\end{Verbatim}


    \begin{Verbatim}[commandchars=\\\{\}]
Performances during training period for JPM
Date Range: 2008-01-01 00:00:00 to 2009-12-31 00:00:00
Sharpe Ratio of Portfolio: 9.254709753508902
Sharpe Ratio of Benchmark : 0.15691840642403027

Cumulative Return of Portfolio: 2.8329
Cumulative Return of Benchmark : 0.012299999999999978

Standard Deviation of Portfolio: 0.00461522467764988
Standard Deviation of Benchmark : 0.017004366271213767

Average Daily Return of Portfolio: 0.0026906386767000376
Average Daily Return of Benchmark : 0.00016808697819094035

Final Portfolio Value: 383290.0
Final Benchmark Value: 101230.0

Portfolio Orders count: 503

    \end{Verbatim}

    \begin{Verbatim}[commandchars=\\\{\}]
/home/emi/anaconda3/lib/python3.6/site-packages/plotly/graph\_objs/\_deprecations.py:396: DeprecationWarning:

plotly.graph\_objs.Margin is deprecated.
Please replace it with one of the following more specific types
  - plotly.graph\_objs.layout.Margin



    \end{Verbatim}

    \begin{center}
    \adjustimage{max size={0.9\linewidth}{0.9\paperheight}}{output_28_2.png}
    \end{center}
    { \hspace*{\fill} \\}
    
    \section{Part 3: Manual Rule-Based
Trader}\label{part-3-manual-rule-based-trader}

In ManualStrategy.py implement a set of rules using the indicators you
created in Part 1 above. Devise some simple logic using your indicators
to enter and exit positions in the stock.

A recommended approach is to create a single logical expression that
yields a -1, 0, or 1, corresponding to a "short," "out" or "long"
position. Example usage this signal: If you are out of the stock, then a
1 would signal a BUY 1000 order. If you are long, a -1 would signal a
SELL 2000 order. You don't have to follow this advice though, so long as
you follow the trading rules outlined above.

You should tweak your rules as best you can to get the best performance
possible during the in sample period (do not peek at out of sample
performance). Use your rule-based strategy to generate an orders
dataframe over the in sample period, then run that dataframe through
your market simulator to create a chart that includes the following
components over the in sample period:

\begin{verbatim}
* Benchmark (see definition above) normalized to 1.0 at the start: Blue line
* Value of the rule-based portfolio (normalized to 1.0 at the start): Black line
* Vertical green lines indicating LONG entry points.
* Vertical red lines indicating SHORT entry points.
\end{verbatim}

    \subsection{Import libraries}\label{import-libraries}

    \begin{Verbatim}[commandchars=\\\{\}]
{\color{incolor}In [{\color{incolor}13}]:} \PY{c+c1}{\PYZsh{} For Manual Rule\PYZhy{}Based Trader}
         \PY{k+kn}{from} \PY{n+nn}{marketsim} \PY{k}{import} \PY{n}{market\PYZus{}simulator}
         \PY{k+kn}{from} \PY{n+nn}{indicators} \PY{k}{import} \PY{n}{get\PYZus{}momentum}\PY{p}{,} \PY{n}{get\PYZus{}sma\PYZus{}indicator}\PY{p}{,} \PY{n}{get\PYZus{}bollinger\PYZus{}bands}\PY{p}{,} \PYZbs{}
         \PY{n}{compute\PYZus{}bollinger\PYZus{}value}\PY{p}{,} \PY{n}{plot\PYZus{}momentum}\PY{p}{,} \PY{n}{plot\PYZus{}sma\PYZus{}indicator}\PY{p}{,} \PY{n}{plot\PYZus{}bollinger}
         \PY{k+kn}{from} \PY{n+nn}{rule\PYZus{}based\PYZus{}strategy} \PY{k}{import} \PY{n}{RuleBasedStrategy}
\end{Verbatim}


    \subsection{Starting cash and symbol of
interest}\label{starting-cash-and-symbol-of-interest}

    \begin{Verbatim}[commandchars=\\\{\}]
{\color{incolor}In [{\color{incolor}14}]:} \PY{n}{start\PYZus{}val} \PY{o}{=} \PY{l+m+mi}{100000}
         \PY{n}{symbol} \PY{o}{=} \PY{l+s+s2}{\PYZdq{}}\PY{l+s+s2}{JPM}\PY{l+s+s2}{\PYZdq{}}
\end{Verbatim}


    \subsection{In-sample performance}\label{in-sample-performance}

Show the performances of portfolio and benchmark in the in-sample
period.

    \begin{Verbatim}[commandchars=\\\{\}]
{\color{incolor}In [{\color{incolor}15}]:} \PY{c+c1}{\PYZsh{} Specify the start and end dates for this period.}
         \PY{n}{start\PYZus{}d} \PY{o}{=} \PY{n}{dt}\PY{o}{.}\PY{n}{datetime}\PY{p}{(}\PY{l+m+mi}{2008}\PY{p}{,} \PY{l+m+mi}{1}\PY{p}{,} \PY{l+m+mi}{1}\PY{p}{)}
         \PY{n}{end\PYZus{}d} \PY{o}{=} \PY{n}{dt}\PY{o}{.}\PY{n}{datetime}\PY{p}{(}\PY{l+m+mi}{2009}\PY{p}{,} \PY{l+m+mi}{12}\PY{p}{,} \PY{l+m+mi}{31}\PY{p}{)}
         
         \PY{c+c1}{\PYZsh{} Get benchmark data}
         \PY{n}{benchmark\PYZus{}prices} \PY{o}{=} \PY{n}{get\PYZus{}data}\PY{p}{(}\PY{p}{[}\PY{n}{symbol}\PY{p}{]}\PY{p}{,} \PY{n}{pd}\PY{o}{.}\PY{n}{date\PYZus{}range}\PY{p}{(}\PY{n}{start\PYZus{}d}\PY{p}{,} \PY{n}{end\PYZus{}d}\PY{p}{)}\PY{p}{,} \PY{n}{addSPY}\PY{o}{=}\PY{k+kc}{False}\PY{p}{)}\PY{o}{.}\PY{n}{dropna}\PY{p}{(}\PY{p}{)}
         
         \PY{c+c1}{\PYZsh{} Create benchmark trades: buy 1000 shares of symbol, hold them till the last date}
         \PY{n}{df\PYZus{}benchmark\PYZus{}trades} \PY{o}{=} \PY{n}{pd}\PY{o}{.}\PY{n}{DataFrame}\PY{p}{(}
                 \PY{n}{data}\PY{o}{=}\PY{p}{[}\PY{p}{(}\PY{n}{benchmark\PYZus{}prices}\PY{o}{.}\PY{n}{index}\PY{o}{.}\PY{n}{min}\PY{p}{(}\PY{p}{)}\PY{p}{,} \PY{n}{symbol}\PY{p}{,} \PY{l+s+s2}{\PYZdq{}}\PY{l+s+s2}{BUY}\PY{l+s+s2}{\PYZdq{}}\PY{p}{,} \PY{l+m+mi}{1000}\PY{p}{)}\PY{p}{,} 
                 \PY{p}{(}\PY{n}{benchmark\PYZus{}prices}\PY{o}{.}\PY{n}{index}\PY{o}{.}\PY{n}{max}\PY{p}{(}\PY{p}{)}\PY{p}{,} \PY{n}{symbol}\PY{p}{,} \PY{l+s+s2}{\PYZdq{}}\PY{l+s+s2}{SELL}\PY{l+s+s2}{\PYZdq{}}\PY{p}{,} \PY{l+m+mi}{1000}\PY{p}{)}\PY{p}{]}\PY{p}{,} 
                 \PY{n}{columns}\PY{o}{=}\PY{p}{[}\PY{l+s+s2}{\PYZdq{}}\PY{l+s+s2}{Date}\PY{l+s+s2}{\PYZdq{}}\PY{p}{,} \PY{l+s+s2}{\PYZdq{}}\PY{l+s+s2}{Symbol}\PY{l+s+s2}{\PYZdq{}}\PY{p}{,} \PY{l+s+s2}{\PYZdq{}}\PY{l+s+s2}{Order}\PY{l+s+s2}{\PYZdq{}}\PY{p}{,} \PY{l+s+s2}{\PYZdq{}}\PY{l+s+s2}{Shares}\PY{l+s+s2}{\PYZdq{}}\PY{p}{]}\PY{p}{)}
         \PY{n}{df\PYZus{}benchmark\PYZus{}trades}\PY{o}{.}\PY{n}{set\PYZus{}index}\PY{p}{(}\PY{l+s+s2}{\PYZdq{}}\PY{l+s+s2}{Date}\PY{l+s+s2}{\PYZdq{}}\PY{p}{,} \PY{n}{inplace}\PY{o}{=}\PY{k+kc}{True}\PY{p}{)}
         
         \PY{c+c1}{\PYZsh{} Create an instance of RuleBasedStrategy}
         \PY{n}{rule\PYZus{}based} \PY{o}{=} \PY{n}{RuleBasedStrategy}\PY{p}{(}\PY{p}{)}
         \PY{n}{df\PYZus{}trades} \PY{o}{=} \PY{n}{rule\PYZus{}based}\PY{o}{.}\PY{n}{test\PYZus{}policy}\PY{p}{(}\PY{n}{symbol}\PY{o}{=}\PY{n}{symbol}\PY{p}{,} \PY{n}{start\PYZus{}date}\PY{o}{=}\PY{n}{start\PYZus{}d}\PY{p}{,} \PY{n}{end\PYZus{}date}\PY{o}{=}\PY{n}{end\PYZus{}d}\PY{p}{)}
         \PY{n+nb}{print} \PY{p}{(}\PY{n}{df\PYZus{}trades}\PY{p}{)}
         
         \PY{c+c1}{\PYZsh{} Retrieve performance stats via a market simulator}
         \PY{n+nb}{print} \PY{p}{(}\PY{l+s+s2}{\PYZdq{}}\PY{l+s+se}{\PYZbs{}n}\PY{l+s+s2}{Performances during training period for }\PY{l+s+si}{\PYZob{}\PYZcb{}}\PY{l+s+s2}{\PYZdq{}}\PY{o}{.}\PY{n}{format}\PY{p}{(}\PY{n}{symbol}\PY{p}{)}\PY{p}{)}
         \PY{n+nb}{print} \PY{p}{(}\PY{l+s+s2}{\PYZdq{}}\PY{l+s+s2}{Date Range: }\PY{l+s+si}{\PYZob{}\PYZcb{}}\PY{l+s+s2}{ to }\PY{l+s+si}{\PYZob{}\PYZcb{}}\PY{l+s+s2}{\PYZdq{}}\PY{o}{.}\PY{n}{format}\PY{p}{(}\PY{n}{start\PYZus{}d}\PY{p}{,} \PY{n}{end\PYZus{}d}\PY{p}{)}\PY{p}{)}
         \PY{n}{orders\PYZus{}count\PYZus{}in}\PY{p}{,} \PY{n}{sharpe\PYZus{}ratio\PYZus{}in}\PY{p}{,} \PY{n}{cum\PYZus{}ret\PYZus{}in}\PY{p}{,} \PY{n}{std\PYZus{}daily\PYZus{}ret\PYZus{}in}\PY{p}{,} \PY{n}{avg\PYZus{}daily\PYZus{}ret\PYZus{}in}\PY{p}{,} \PY{n}{final\PYZus{}value\PYZus{}in} \PY{o}{=} \PY{n}{market\PYZus{}simulator}\PY{p}{(}\PY{n}{df\PYZus{}trades}\PY{p}{,} \PY{n}{df\PYZus{}benchmark\PYZus{}trades}\PY{p}{,} \PY{n}{start\PYZus{}val}\PY{o}{=}\PY{n}{start\PYZus{}val}\PY{p}{,} \PY{n}{vertical\PYZus{}lines}\PY{o}{=}\PY{k+kc}{True}\PY{p}{)}
\end{Verbatim}


    \begin{Verbatim}[commandchars=\\\{\}]
/home/emi/Jupyter/ML4T\_2018Spring/manual\_strategy/indicators.py:100: FutureWarning:

pd.rolling\_apply is deprecated for Series and will be removed in a future version, replace with 
	Series.rolling(window=14,center=False).apply(func=<function>,args=<tuple>,kwargs=<dict>)


    \end{Verbatim}

    \begin{Verbatim}[commandchars=\\\{\}]
           Symbol Order  Shares
Date                           
2008-03-27    JPM  SELL    1000
2009-06-02    JPM   BUY    1000
2009-06-03    JPM   BUY    1000
2009-12-31    JPM  SELL    1000

Performances during training period for JPM
Date Range: 2008-01-01 00:00:00 to 2009-12-31 00:00:00
Sharpe Ratio of Portfolio: 0.4085504721694376
Sharpe Ratio of Benchmark : 0.1533866908360293

Cumulative Return of Portfolio: 0.13203426439620203
Cumulative Return of Benchmark : 0.010236207848477452

Standard Deviation of Portfolio: 0.01541790995682459
Standard Deviation of Benchmark : 0.017041225678409405

Average Daily Return of Portfolio: 0.00039679934935582853
Average Daily Return of Benchmark : 0.00016466004720283744

Final Portfolio Value: 112970.84999999999
Final Benchmark Value: 100819.25

Portfolio Orders count: 447

    \end{Verbatim}

    \begin{Verbatim}[commandchars=\\\{\}]
/home/emi/anaconda3/lib/python3.6/site-packages/plotly/graph\_objs/\_deprecations.py:396: DeprecationWarning:

plotly.graph\_objs.Margin is deprecated.
Please replace it with one of the following more specific types
  - plotly.graph\_objs.layout.Margin



    \end{Verbatim}

    \begin{center}
    \adjustimage{max size={0.9\linewidth}{0.9\paperheight}}{output_35_3.png}
    \end{center}
    { \hspace*{\fill} \\}
    
    \subsubsection{Description}\label{description}

\textbf{Indicators}

\begin{verbatim}
1. Momentum: With a window=14
2. RSI (Relative Strengh Index)  
3. SMA (Simple Moving Average: With a window=30
\end{verbatim}

\textbf{Signals}

\begin{verbatim}
1. IF momentum < -0.07 then SELL.
2. If momentum > 0.14 then BUY.
3. If rsi_indicator < 50 then BUY.
4. If rsi_indicator > 50 then SELL.
5. if sma_indicator < 0.0 then BUY.
6. If sma_indicator > 0.0 then SELL.
\end{verbatim}

\textbf{Rules}

We should BUY if momentum and RSI and SMA give BUY signals. We should
SELL if momentum and RSI and SMA give SELL signals.

\textbf{Chart}

You can see the points of LONG entries, (red vertical lines) (BUY) and
the SHORT ones (green vertical lines) (SELL)

\textbf{Results}

As you can see in the statistics above, our rule-based strategy have
outperformed the benchmark over the in sample period.

Final Portfolio Value: 112970.84 Final Benchmark Value: 100819.25

    \section{Part 4: Comparative
Analysis}\label{part-4-comparative-analysis}

Evaluate the performance of your strategy in the out of sample period.
Note that you should not train or tweak your approach on this data. You
should use the classification learned using the in sample data only.
Create a chart that shows, out of sample:

\begin{verbatim}
* Benchmark (see definition above) normalized to 1.0 at the start: Blue line
* Performance of manual strategy: Black line
* Both should be normalized to 1.0 at the start.
\end{verbatim}

Create a table that summarizes the performance of the stock, and the
manual strategy for both in sample and out of sample periods. Explain
WHY these differences occur.

    \begin{Verbatim}[commandchars=\\\{\}]
{\color{incolor}In [{\color{incolor}16}]:} \PY{c+c1}{\PYZsh{} For Manual Rule\PYZhy{}Based Trader}
         \PY{k+kn}{from} \PY{n+nn}{marketsim} \PY{k}{import} \PY{n}{market\PYZus{}simulator}
         \PY{k+kn}{from} \PY{n+nn}{indicators} \PY{k}{import} \PY{n}{get\PYZus{}momentum}\PY{p}{,} \PY{n}{get\PYZus{}sma\PYZus{}indicator}\PY{p}{,} \PY{n}{get\PYZus{}bollinger\PYZus{}bands}\PY{p}{,} \PYZbs{}
         \PY{n}{compute\PYZus{}bollinger\PYZus{}value}\PY{p}{,} \PY{n}{plot\PYZus{}momentum}\PY{p}{,} \PY{n}{plot\PYZus{}sma\PYZus{}indicator}\PY{p}{,} \PY{n}{plot\PYZus{}bollinger}\PY{p}{,} \PY{n}{plot\PYZus{}performance}
         \PY{k+kn}{from} \PY{n+nn}{rule\PYZus{}based\PYZus{}strategy} \PY{k}{import} \PY{n}{RuleBasedStrategy}
\end{Verbatim}


    \subsection{Starting cash and symbol of
interest}\label{starting-cash-and-symbol-of-interest}

    \begin{Verbatim}[commandchars=\\\{\}]
{\color{incolor}In [{\color{incolor}17}]:} \PY{n}{start\PYZus{}val} \PY{o}{=} \PY{l+m+mi}{100000}
         \PY{n}{symbol} \PY{o}{=} \PY{l+s+s2}{\PYZdq{}}\PY{l+s+s2}{JPM}\PY{l+s+s2}{\PYZdq{}}
\end{Verbatim}


    \subsection{Out of sample performance}\label{out-of-sample-performance}

Show the performances of portfolio and benchmark in the out of sample
period.

    \begin{Verbatim}[commandchars=\\\{\}]
{\color{incolor}In [{\color{incolor}18}]:} \PY{c+c1}{\PYZsh{} Specify the start and end dates for this period.}
         \PY{n}{start\PYZus{}d} \PY{o}{=} \PY{n}{dt}\PY{o}{.}\PY{n}{datetime}\PY{p}{(}\PY{l+m+mi}{2010}\PY{p}{,} \PY{l+m+mi}{1}\PY{p}{,} \PY{l+m+mi}{1}\PY{p}{)}
         \PY{n}{end\PYZus{}d} \PY{o}{=} \PY{n}{dt}\PY{o}{.}\PY{n}{datetime}\PY{p}{(}\PY{l+m+mi}{2011}\PY{p}{,} \PY{l+m+mi}{12}\PY{p}{,} \PY{l+m+mi}{31}\PY{p}{)}
         
         \PY{c+c1}{\PYZsh{} Get benchmark data}
         \PY{n}{benchmark\PYZus{}prices} \PY{o}{=} \PY{n}{get\PYZus{}data}\PY{p}{(}\PY{p}{[}\PY{n}{symbol}\PY{p}{]}\PY{p}{,} \PY{n}{pd}\PY{o}{.}\PY{n}{date\PYZus{}range}\PY{p}{(}\PY{n}{start\PYZus{}d}\PY{p}{,} \PY{n}{end\PYZus{}d}\PY{p}{)}\PY{p}{,} \PY{n}{addSPY}\PY{o}{=}\PY{k+kc}{False}\PY{p}{)}\PY{o}{.}\PY{n}{dropna}\PY{p}{(}\PY{p}{)}
         
         \PY{c+c1}{\PYZsh{} Create benchmark trades: buy 1000 shares of symbol, hold them till the last date}
         \PY{n}{df\PYZus{}benchmark\PYZus{}trades} \PY{o}{=} \PY{n}{pd}\PY{o}{.}\PY{n}{DataFrame}\PY{p}{(}
                 \PY{n}{data}\PY{o}{=}\PY{p}{[}\PY{p}{(}\PY{n}{benchmark\PYZus{}prices}\PY{o}{.}\PY{n}{index}\PY{o}{.}\PY{n}{min}\PY{p}{(}\PY{p}{)}\PY{p}{,} \PY{n}{symbol}\PY{p}{,} \PY{l+s+s2}{\PYZdq{}}\PY{l+s+s2}{BUY}\PY{l+s+s2}{\PYZdq{}}\PY{p}{,} \PY{l+m+mi}{1000}\PY{p}{)}\PY{p}{,} 
                 \PY{p}{(}\PY{n}{benchmark\PYZus{}prices}\PY{o}{.}\PY{n}{index}\PY{o}{.}\PY{n}{max}\PY{p}{(}\PY{p}{)}\PY{p}{,} \PY{n}{symbol}\PY{p}{,} \PY{l+s+s2}{\PYZdq{}}\PY{l+s+s2}{SELL}\PY{l+s+s2}{\PYZdq{}}\PY{p}{,} \PY{l+m+mi}{1000}\PY{p}{)}\PY{p}{]}\PY{p}{,} 
                 \PY{n}{columns}\PY{o}{=}\PY{p}{[}\PY{l+s+s2}{\PYZdq{}}\PY{l+s+s2}{Date}\PY{l+s+s2}{\PYZdq{}}\PY{p}{,} \PY{l+s+s2}{\PYZdq{}}\PY{l+s+s2}{Symbol}\PY{l+s+s2}{\PYZdq{}}\PY{p}{,} \PY{l+s+s2}{\PYZdq{}}\PY{l+s+s2}{Order}\PY{l+s+s2}{\PYZdq{}}\PY{p}{,} \PY{l+s+s2}{\PYZdq{}}\PY{l+s+s2}{Shares}\PY{l+s+s2}{\PYZdq{}}\PY{p}{]}\PY{p}{)}
         \PY{n}{df\PYZus{}benchmark\PYZus{}trades}\PY{o}{.}\PY{n}{set\PYZus{}index}\PY{p}{(}\PY{l+s+s2}{\PYZdq{}}\PY{l+s+s2}{Date}\PY{l+s+s2}{\PYZdq{}}\PY{p}{,} \PY{n}{inplace}\PY{o}{=}\PY{k+kc}{True}\PY{p}{)}
         
         \PY{c+c1}{\PYZsh{} Create an instance of RuleBasedStrategy}
         \PY{n}{rule\PYZus{}based} \PY{o}{=} \PY{n}{RuleBasedStrategy}\PY{p}{(}\PY{p}{)}
         \PY{n}{df\PYZus{}trades} \PY{o}{=} \PY{n}{rule\PYZus{}based}\PY{o}{.}\PY{n}{test\PYZus{}policy}\PY{p}{(}\PY{n}{symbol}\PY{o}{=}\PY{n}{symbol}\PY{p}{,} \PY{n}{start\PYZus{}date}\PY{o}{=}\PY{n}{start\PYZus{}d}\PY{p}{,} \PY{n}{end\PYZus{}date}\PY{o}{=}\PY{n}{end\PYZus{}d}\PY{p}{)}
         \PY{n+nb}{print} \PY{p}{(}\PY{n}{df\PYZus{}trades}\PY{p}{)}
         
         \PY{c+c1}{\PYZsh{} Retrieve performance stats via a market simulator}
         \PY{n+nb}{print} \PY{p}{(}\PY{l+s+s2}{\PYZdq{}}\PY{l+s+se}{\PYZbs{}n}\PY{l+s+s2}{Performances during testing period for }\PY{l+s+si}{\PYZob{}\PYZcb{}}\PY{l+s+s2}{\PYZdq{}}\PY{o}{.}\PY{n}{format}\PY{p}{(}\PY{n}{symbol}\PY{p}{)}\PY{p}{)}
         \PY{n+nb}{print} \PY{p}{(}\PY{l+s+s2}{\PYZdq{}}\PY{l+s+s2}{Date Range: }\PY{l+s+si}{\PYZob{}\PYZcb{}}\PY{l+s+s2}{ to }\PY{l+s+si}{\PYZob{}\PYZcb{}}\PY{l+s+s2}{\PYZdq{}}\PY{o}{.}\PY{n}{format}\PY{p}{(}\PY{n}{start\PYZus{}d}\PY{p}{,} \PY{n}{end\PYZus{}d}\PY{p}{)}\PY{p}{)}
         \PY{n}{orders\PYZus{}count\PYZus{}out}\PY{p}{,} \PY{n}{sharpe\PYZus{}ratio\PYZus{}out}\PY{p}{,} \PY{n}{cum\PYZus{}ret\PYZus{}out}\PY{p}{,} \PY{n}{std\PYZus{}daily\PYZus{}ret\PYZus{}out}\PY{p}{,} \PY{n}{avg\PYZus{}daily\PYZus{}ret\PYZus{}out}\PY{p}{,} \PY{n}{final\PYZus{}value\PYZus{}out} \PY{o}{=} \PY{n}{market\PYZus{}simulator}\PY{p}{(}\PY{n}{df\PYZus{}trades}\PY{p}{,} \PY{n}{df\PYZus{}benchmark\PYZus{}trades}\PY{p}{,} \PY{n}{start\PYZus{}val}\PY{o}{=}\PY{n}{start\PYZus{}val}\PY{p}{,} \PY{n}{vertical\PYZus{}lines}\PY{o}{=}\PY{k+kc}{True}\PY{p}{)}
\end{Verbatim}


    \begin{Verbatim}[commandchars=\\\{\}]
/home/emi/Jupyter/ML4T\_2018Spring/manual\_strategy/indicators.py:100: FutureWarning:

pd.rolling\_apply is deprecated for Series and will be removed in a future version, replace with 
	Series.rolling(window=14,center=False).apply(func=<function>,args=<tuple>,kwargs=<dict>)


    \end{Verbatim}

    \begin{Verbatim}[commandchars=\\\{\}]
           Symbol Order  Shares
Date                           
2010-06-25    JPM  SELL    1000
2011-12-30    JPM   BUY    1000

Performances during testing period for JPM
Date Range: 2010-01-01 00:00:00 to 2011-12-31 00:00:00
Sharpe Ratio of Portfolio: 0.3146236665412426
Sharpe Ratio of Benchmark : -0.26361722846060553

Cumulative Return of Portfolio: 0.04967247568304156
Cumulative Return of Benchmark : -0.08530881679439029

Standard Deviation of Portfolio: 0.00799402183930532
Standard Deviation of Benchmark : 0.00850128395312379

Average Daily Return of Portfolio: 0.00015843694070154216
Average Daily Return of Benchmark : -0.00014117507975314027

Final Portfolio Value: 104759.15
Final Benchmark Value: 91273.1

Portfolio Orders count: 384

    \end{Verbatim}

    \begin{Verbatim}[commandchars=\\\{\}]
/home/emi/anaconda3/lib/python3.6/site-packages/plotly/graph\_objs/\_deprecations.py:396: DeprecationWarning:

plotly.graph\_objs.Margin is deprecated.
Please replace it with one of the following more specific types
  - plotly.graph\_objs.layout.Margin



    \end{Verbatim}

    \begin{center}
    \adjustimage{max size={0.9\linewidth}{0.9\paperheight}}{output_42_3.png}
    \end{center}
    { \hspace*{\fill} \\}
    
    \subsection{Stock Performance table}\label{stock-performance-table}

Create a table that summarizes the performance of the stock, and the
manual strategy for both in sample and out of sample periods. Explain
WHY these differences occur.

    \subsubsection{Create a dataframe of in-sample and out sample portfolio
data}\label{create-a-dataframe-of-in-sample-and-out-sample-portfolio-data}

    \begin{Verbatim}[commandchars=\\\{\}]
{\color{incolor}In [{\color{incolor}19}]:} \PY{c+c1}{\PYZsh{} Dataframe columns}
         \PY{n}{columns} \PY{o}{=} \PY{p}{[}\PY{l+s+s1}{\PYZsq{}}\PY{l+s+s1}{Indicator}\PY{l+s+s1}{\PYZsq{}}\PY{p}{,}\PY{l+s+s1}{\PYZsq{}}\PY{l+s+s1}{In\PYZhy{}sample}\PY{l+s+s1}{\PYZsq{}}\PY{p}{,} \PY{l+s+s1}{\PYZsq{}}\PY{l+s+s1}{Out of sample}\PY{l+s+s1}{\PYZsq{}}\PY{p}{]}
         
         \PY{c+c1}{\PYZsh{} Create and add data to dataframe}
         \PY{n}{perform\PYZus{}df} \PY{o}{=} \PY{n}{pd}\PY{o}{.}\PY{n}{DataFrame}\PY{p}{(}\PY{n}{columns}\PY{o}{=}\PY{n}{columns}\PY{p}{)}
         \PY{n}{perform\PYZus{}df}\PY{o}{.}\PY{n}{loc}\PY{p}{[}\PY{n+nb}{len}\PY{p}{(}\PY{n}{perform\PYZus{}df}\PY{p}{)}\PY{p}{]} \PY{o}{=} \PY{p}{[}\PY{l+s+s2}{\PYZdq{}}\PY{l+s+s2}{Orders count}\PY{l+s+s2}{\PYZdq{}}\PY{p}{,} \PY{n}{orders\PYZus{}count\PYZus{}in}\PY{p}{,} \PY{n}{orders\PYZus{}count\PYZus{}out}\PY{p}{]}
         \PY{n}{perform\PYZus{}df}\PY{o}{.}\PY{n}{loc}\PY{p}{[}\PY{n+nb}{len}\PY{p}{(}\PY{n}{perform\PYZus{}df}\PY{p}{)}\PY{p}{]} \PY{o}{=} \PY{p}{[}\PY{l+s+s2}{\PYZdq{}}\PY{l+s+s2}{Sharpe Ratio}\PY{l+s+s2}{\PYZdq{}}\PY{p}{,} \PY{n}{sharpe\PYZus{}ratio\PYZus{}in}\PY{p}{,} \PY{n}{sharpe\PYZus{}ratio\PYZus{}out}\PY{p}{]}
         \PY{n}{perform\PYZus{}df}\PY{o}{.}\PY{n}{loc}\PY{p}{[}\PY{n+nb}{len}\PY{p}{(}\PY{n}{perform\PYZus{}df}\PY{p}{)}\PY{p}{]} \PY{o}{=} \PY{p}{[}\PY{l+s+s2}{\PYZdq{}}\PY{l+s+s2}{Cumulative Return}\PY{l+s+s2}{\PYZdq{}}\PY{p}{,} \PY{n}{cum\PYZus{}ret\PYZus{}in}\PY{p}{,} \PY{n}{cum\PYZus{}ret\PYZus{}out}\PY{p}{]}
         \PY{n}{perform\PYZus{}df}\PY{o}{.}\PY{n}{loc}\PY{p}{[}\PY{n+nb}{len}\PY{p}{(}\PY{n}{perform\PYZus{}df}\PY{p}{)}\PY{p}{]} \PY{o}{=} \PY{p}{[}\PY{l+s+s2}{\PYZdq{}}\PY{l+s+s2}{Standard Deviation}\PY{l+s+s2}{\PYZdq{}}\PY{p}{,} \PY{n}{std\PYZus{}daily\PYZus{}ret\PYZus{}in}\PY{p}{,} \PY{n}{std\PYZus{}daily\PYZus{}ret\PYZus{}out}\PY{p}{]}
         \PY{n}{perform\PYZus{}df}\PY{o}{.}\PY{n}{loc}\PY{p}{[}\PY{n+nb}{len}\PY{p}{(}\PY{n}{perform\PYZus{}df}\PY{p}{)}\PY{p}{]} \PY{o}{=} \PY{p}{[}\PY{l+s+s2}{\PYZdq{}}\PY{l+s+s2}{Average Daily Return}\PY{l+s+s2}{\PYZdq{}}\PY{p}{,} \PY{n}{avg\PYZus{}daily\PYZus{}ret\PYZus{}in}\PY{p}{,} \PY{n}{avg\PYZus{}daily\PYZus{}ret\PYZus{}out}\PY{p}{]}
         \PY{n}{perform\PYZus{}df}\PY{o}{.}\PY{n}{loc}\PY{p}{[}\PY{n+nb}{len}\PY{p}{(}\PY{n}{perform\PYZus{}df}\PY{p}{)}\PY{p}{]} \PY{o}{=} \PY{p}{[}\PY{l+s+s2}{\PYZdq{}}\PY{l+s+s2}{Final Portfolio Value}\PY{l+s+s2}{\PYZdq{}}\PY{p}{,} \PY{n}{final\PYZus{}value\PYZus{}in}\PY{p}{,} \PY{n}{final\PYZus{}value\PYZus{}out}\PY{p}{]}
         
         \PY{n}{pedf} \PY{o}{=} \PY{n}{perform\PYZus{}df}\PY{o}{.}\PY{n}{set\PYZus{}index}\PY{p}{(}\PY{l+s+s2}{\PYZdq{}}\PY{l+s+s2}{Indicator}\PY{l+s+s2}{\PYZdq{}}\PY{p}{)}
         \PY{n}{pedf}
\end{Verbatim}


\begin{Verbatim}[commandchars=\\\{\}]
{\color{outcolor}Out[{\color{outcolor}19}]:}                          In-sample Out of sample
         Indicator                                       
         Orders count                   447           384
         Sharpe Ratio               0.40855      0.314624
         Cumulative Return         0.132034     0.0496725
         Standard Deviation       0.0154179    0.00799402
         Average Daily Return   0.000396799   0.000158437
         Final Portfolio Value       112971        104759
\end{Verbatim}
            
    \subsubsection{Plot In-sample and Out of sample
performances}\label{plot-in-sample-and-out-of-sample-performances}

    \begin{Verbatim}[commandchars=\\\{\}]
{\color{incolor}In [{\color{incolor}20}]:} \PY{n}{plot\PYZus{}performance}\PY{p}{(}\PY{n}{pedf}\PY{p}{)}
\end{Verbatim}


    \begin{center}
    \adjustimage{max size={0.9\linewidth}{0.9\paperheight}}{output_47_0.png}
    \end{center}
    { \hspace*{\fill} \\}
    
    \section{Report}\label{report}

According to all


    % Add a bibliography block to the postdoc
    
    
    
    \end{document}
